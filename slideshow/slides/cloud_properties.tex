\section{Cloud properties}

\subsection{Health checks}
\begin{frame}[containsverbatim]
	\frametitle{Health checks}
	
	\begin{itemize}
		\item It is a Docker instruction to determine the state of a container (\emph{healthy}/\emph{unhealthy}).
		\item Health checks are performed at regular intervals.
	\end{itemize}
	
	\begin{figure}[h]
		\begin{minted}[tabsize=2, fontsize=\small]{yaml}
      healthcheck:
        test: ["CMD", "redis-cli", "ping"]
        interval: 30s
        timeout: 10s
        retries: 3
		\end{minted}
		\caption{Health check definition for Redis container}
	\end{figure}
\end{frame}

\begin{frame}[containsverbatim]
	\frametitle{Health checks}
	
	Health checks aim to provide
	\begin{itemize}
		\item \emph{Reliability}: they monitor that an application is working properly;
		\item \emph{Resilience} and \emph{Self-healing}: it is possible to detect a fault and reboot the service.
	\end{itemize}
	
	\begin{figure}[h]
		\begin{minted}[tabsize=2, fontsize=\small, breaklines]{yaml}
    healthcheck:
      test: ["CMD", "curl", "--fail", "http://localhost:5000/health"]
      interval: 30s
      timeout: 10s
      retries: 3
		\end{minted}
		\caption{Health check definition for Decoding container}
	\end{figure}
\end{frame}

\subsection{Restart policies}
\begin{frame}[containsverbatim]
	\frametitle{Restart policies}
	
	\begin{itemize}
		\item They define how Docker should handle containers when they stop or crash.
		\item This policies are important to guarantee \emph{reliability} and \emph{high availability}.
		\item For each container the policy \mintinline{yaml}|restart: always| is set, this means that the service is always restarted regardless the cause stopping it.
	\end{itemize}
	
	\begin{figure}[h]
		\begin{minted}[tabsize=2, fontsize=\small, breaklines]{yaml}
        mongodb:
          image: mongo:latest
          container_name: mongodb
          restart: always
		\end{minted}
		\caption{Definition of the restart policy for MongoDB}
	\end{figure}
\end{frame}