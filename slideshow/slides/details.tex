\section{Detailed view}

\subsection{Data Generator}
\begin{frame}
	\frametitle{Data Generator}
	
	\begin{itemize}
		\item It simulates data produced by each power meter.
		\item It publish data on a specific Redis channel.\\
		E.g. Data from \texttt{machine1} of \texttt{customer1} are published on \texttt{data.customer1.machine1}
	\end{itemize}
	
	\bigskip
	\begin{figure}[h]
		\centering
		\includegraphics[width=0.6\textwidth]{./drawings/MachinesHierarchy.pdf}
	\end{figure}
\end{frame}


\subsection{Redis}
\begin{frame}
	\frametitle{Redis}
	
	\begin{itemize}
		\item It is a key-value database that offers the functionalities of a message broker.
		\item It receives data from all the meters and forward them to two microservices: \emph{DataDecoding} and \emph{Storeroom}.
		\item It is the entry point of the entire system. It exposes the port 6379.
	\end{itemize}
	
	\begin{figure}[h]
		\centering
		\includegraphics[width=0.7\textwidth]{./img/redis-publish-subscriber.png}
	\end{figure}
\end{frame}

\begin{frame}[containsverbatim]
	\frametitle{Redis}
	
	\begin{itemize}
		\item Publishers need to authenticate before starting to send data.
		\item A list of users is defined inside an \emph{Access Control List} (ACL).
		\item Each user can perform only a subset of commands.
	\end{itemize}
	
	\begin{figure}[h]
		\begin{minted}[tabsize=2, fontsize=\small]{text}
			user default off
			user customer on >customerpass +PUBLISH &data.*
			user admin on >admin ~* &* +@all
			user decoding on >dec +PSUBSCRIBE &data.*
			user storeroom on >store +PSUBSCRIBE &data.*
		\end{minted}
		\caption{File \texttt{redis.conf} specifing the ACL}
	\end{figure}
	
	\begin{figure}[h]
		\begin{minted}[tabsize=2, fontsize=\small, breaklines]{python}
			r = redis.Redis(host='redis', port=6379, username='decoding', password='dec', decode_responses=True)
		\end{minted}
		\caption{Connection to Redis server in a Python script}
	\end{figure}
\end{frame}


\subsection{Data decoding}
\begin{frame}[containsverbatim]
	\frametitle{Data decoding}
	
	\begin{itemize}
		\item It authenticates in Redis and subscribes to channel '\verb|data.*|'.
		\item It receives a hexadecimal string that is split to obtain data, hour and numeric value of the measure.\\
		E.g. \verb|'2024-09-07T19:53:19.561339'| $\rightarrow$ \verb|'1268b41553b7'|\\
		Numeric value \verb|68 DEC| $\rightarrow$ \verb|44 HEX|.\\
		Then the two strings are concatenated: \verb|'1268b41553b744'|.
		\item It prepares a JSON document and writes it into the database.
	\end{itemize}
	
	\begin{figure}[h]
		\centering
		\begin{minted}[tabsize=2, fontsize=\small]{json}
			{
				"customer": "customer1",
				"machine": "machine1",
				"date": "2024-07-18T18:05:05Z",
				"EE": 32
			}
		\end{minted}
		\caption{Example of document}
	\end{figure}
	
\end{frame}


\subsection{MongoDB Database}
\begin{frame}[containsverbatim]
	\frametitle{MongoDB Database}
	
	\begin{itemize}
		\item It stores all measurements.
		\item Clients use a specific user called \texttt{'client-1'} with only read and write permissions.
		\item DB and users are defined in a JavaScript file executed at the start-up of container.
	\end{itemize}
	
	\begin{figure}[h]
		\begin{minted}[tabsize=2, fontsize=\small]{js}
			db = db.getSiblingDB('cct')
			db.createCollection('measurements')
			db.createUser({
				user: "client-1",
				pwd:  "client-1",
				roles:[{role: "readWrite" , db:"cct"}]
			})
		\end{minted}
		\caption{Configuration file of MongoDB}
	\end{figure}
	
	%	\begin{figure}[h]
		%		\centering
		%		\includegraphics[width=0.3\textwidth]{./img/MongoDB-Logo.png}
		%	\end{figure}
	
\end{frame}


\subsection{Store connector}
\begin{frame}[containsverbatim]
	\frametitle{Store connector}
	
	\begin{itemize}
		\item It subscribes to Redis channel '\verb|data.*|'.
		\item It encapsulates raw data (hexadecimal strings) into text files.
		\item A unique filename is set to every file, given following this scheme
		\mintinline{python}|f"{self.machine_name}_{current_time}_{unique_id}"|,
		where \mintinline{python}|unique_id = uuid.uuid4()|.
		\item Files are sent for permanent storage.
	\end{itemize}
	
\end{frame}

\subsection{MinIO Object Store}
\begin{frame}
	\frametitle{MinIO Object Store}
	
	\begin{itemize}
		\item It is an object store platform, organized in buckets and fully compatible with Amazon S3.
		\item It takes care of the permanent storage of raw data.
		\item One bucket for each customer.
	\end{itemize}
	
	\begin{figure}[h]
		\centering
		\includegraphics[width=0.5\textwidth]{./img/buckets.png}
	\end{figure}
\end{frame}